% !TEX encoding = UTF-8 Unicode
\documentclass[a4paper]{article}
\usepackage[bulgarian]{babel}
\usepackage[utf8]{inputenc}
\usepackage{amssymb, amsthm, amsmath, mathrsfs,alltt}
\usepackage{latexsym}
\usepackage{enumerate}
\usepackage{datetime}
%this package allows for hyperlinks within the pdf document
\usepackage[colorlinks=true, linkcolor=blue,pdfstartview=FitV,
citecolor=green, urlcolor=blue]{hyperref}

\theoremstyle{definition}
\newtheorem{theorem}{Теорема}
\newtheorem{lemma}{Лема}
\newtheorem{corollary}{Следствие}
\newtheorem{proposition}{Твърдение}
\newtheorem{example}{Пример}
\newtheorem{definition}{Определение}
\newtheorem{question}{Въпрос}
\newtheorem{problem}{Задача}
\newtheorem{remark}{Забележка}

\newenvironment{hint}{\noindent{\bf Упътване.}\hspace*{1em}}{\qed\par\vspace*{1em}}

\usepackage{mysymbols}
\reversemarginpar
\usepackage{marginnote}

% \let\oldmarginpar\marginpar
% \renewcommand\marginpar[1]{\leavevmode\oldmarginpar{\raggedright\scriptsize #1}}



% \author{Стефан Вътев\thanks{Draft compiled at \currenttime\ on \today}}
\title{М.Е.С.И.}
% \institute{
% Софийски Университет, Факултет по математика и информатика,\\ 
% бул. Джеймс Баучер 5, 1164 София, България\\ 
% \email{stefanv@fmi.uni-sofia.bg}
% }
\begin{document}
\maketitle

\section{Увод}

\begin{itemize}
\item
  Машини на Тюринг
\item
  Разрешими и полуразрешими езици
\item
  Сложностни класове $\texttt{PSPACE}$, $\texttt{NPSPACE}$, $\texttt{P}$, $\texttt{NP}$ и други.
\item
  Сводимости между проблеми: сводимост на Карп $\leq_P$ и ефективна сводимост $\leq_m$.
\end{itemize}

\begin{theorem}
  \marginnote{\scriptsize \cite[стр. 47]{papadimitriou}}
  Ако $L$ се разрешава от недетерминистичната машина на Тюринг $\N$ за време $f(n)$, то
  $L$ се разрешава от трилентова детерминистична машина на Тюринг $\M$ за време $\mathcal{O}(c^{f(n)})$, където $c$ зависи от $\N$.
\end{theorem}

\section{Двупосочни автомати или защо писането е важно}
\begin{definition}
Двапосочен автомат е машина на Тюринг $M=\langle \Sigma,\Gamma,Q,s,\Delta,F,B\rangle$, за която
$\$,\#\in \Gamma\setminus\Sigma$ и  всеки преход е от вида $t=\langle (p,a),(q,a,d)\rangle$, тоест буквите на лентата не се променят.
\end{definition}
\begin{definition}
Език на двупосочен автомат $M$ е:
\begin{equation*}
L_{2way}(M)=\{\alpha \in \Sigma^*\,|\,\exists f\in F( \langle \varepsilon,s,\#\alpha\$\rangle \Rightarrow_M^* \langle \#\alpha\$,f,\varepsilon\rangle\}.
\end{equation*}
\end{definition}
\begin{theorem}
Класът от езиците, разпознавани от двупосочни автомати е точно класът от регулярни езици.
\end{theorem}

\section{Критерий за разрешимост}

\marginnote{\scriptsize \cite{hopcroft}}
Нека $\Ss$ е множество от полуразрешими езици над фиксирана азбука $\Sigma$.
Например, 
\[\Ss = \{L \subseteq \Sigma^\star \mid L\text{ е регулярен език}\}.\]
Ще казваме, че $\Ss$ е свойство на полуразрешимите езици.
$\Ss$ е {\bf тривиално свойство}, ако $\Ss = \emptyset$ или $\Ss$ съдържа точно всички полуразрешими езици.
Нека разгледаме изброимото множество от машини на Тюринг, които разпознават езиците от $\Ss$.
Ще представим това множество като език от кодовете на тези машини на Тюринг, т.е.
\[\texttt{Ind}(\Ss) = \{\code{\M} \mid \text{$\M$ е машина на Тюринг и } \L(\M) \in \Ss\}.\]

\begin{theorem}[Райс]
  \index{Райс}
  \marginnote{\scriptsize \cite[стр. 62]{papadimitriou}}
  За всяко нетривиално свойство $\Ss$ на полуразрешимите езици, $\texttt{Ind}(\Ss)$ е неразрешимо.
\end{theorem}


\section{Критерий за полуразрешимост}

\marginnote{\scriptsize \cite{hopcroft}}
\begin{lemma}
  Нека $\Ss$ е свойство на полуразрешимите езици.
  Ако съществува безкраен език $L_0 \in \Ss$, който няма крайно подмножество в $\Ss$,
  то $\texttt{Ind}(\Ss)$ не е полуразрешим език.  
\end{lemma}

\begin{lemma}
  Нека $L_1$ е език в $\Ss$ и нека $L_2$ е полуразрешим език, като $L_1 \subset L_2$ и $L_2 \not\in\Ss$.
  Тогава $\texttt{Ind}(\Ss)$ не е полуразрешим език.
\end{lemma}

\begin{theorem}[Райс-Шапиро]
  Нека $\texttt{Ind}(\Ss)$ е полуразрешим език. Тогава е изпълнено, че:
  \[L \in \texttt{Ind}(\Ss) \iff (\forall L_0)[L_0\text{ е краен и }L_0 \subseteq L \implies L_0 \in \Ss].\]
\end{theorem}

\section{Проблем на Пост за съответствието}

\begin{lemma}
  \marginnote{\scriptsize \cite[стр. 193]{hopcroft}}
  Съществува алгоритъм, който свежда MPCP към PCP.
\end{lemma}

\begin{theorem}[Пост]
  Проблемът за съответствието на Пост е неразрешим при азбука $\Sigma$ с поне два символа.
\end{theorem}

\begin{proposition}
  Езикът
  \[L = \{\code{G} \mid G \text{ е еднозначна безконтекстна граматика}\}\]
  е неразрешим.
\end{proposition}

\begin{proposition}
  Езикът
  \[L = \{\code{G_1}\sharp\code{G_2} \mid \mathcal{L}(G_1) \cap \mathcal{L}(G_2) \neq \emptyset\}\]
  е неразрешим.
\end{proposition}


\section{Неразрешими езици}

\begin{proposition}
  Езикът
  \[L = \{\code{G_1}\sharp\code{G_2} \mid \mathcal{L}(G_1) \cap \mathcal{L}(G_2) = \emptyset\}\]
  не е полуразрешим.
\end{proposition}

\begin{proposition}
  Езикът
  \[L = \{\code{G} \mid \L(G) = \Sigma^\star\}\]
  не е полуразрешим.
\end{proposition}

\begin{corollary}
  Езикът
  \[L = \{\code{G_1}\sharp\code{G_2} \mid \mathcal{L}(G_1) = \mathcal{L}(G_2)\}\]
  не е полуразрешим.
\end{corollary}

\begin{theorem}[Грейбах]
  \marginnote{\scriptsize \cite[стр. 205]{hopcroft}}
  Нека $\mathcal{C}$ е клас от езици, за който съществува ефективно кодиране $\code{L}$ на езиците в $\mathcal{C}$ и който е:
  \begin{itemize}
  \item 
    ефективно затворен относно обединение;
  \item
    ефективно затворен относно конкатенация с регулярен език;
  \item
    "$= \Sigma^\star$" е неразрешим за достатъчно голяма $\Sigma$.
  \end{itemize}
  Нека $P$ е нетривиално свойство на $\mathcal{C}$, което е изпълнено за всеки регулярен език и ако $L \in P$,
  то $L/_a \in P$, където
  \[L/_a = \{\omega \mid \omega a \in L\}.\]
  Тогава езикът $\{\code{L} \mid P(L)\ \&\ L \in \mathcal{C}\}$ е неразрешим.
\end{theorem}

\begin{corollary}
  Езикът
  \[L = \{\code{G} \mid \L(G) \text{ е регулярен}\}\]
  не е полуразрешим.
\end{corollary}

\section{Контекстносвободни езици}
\begin{theorem}
Класът на контекстносвободните езици е затворен относно:
\begin{enumerate}
\item сечение с регулярни езици.
\item образи на хомоморфизми.
\item образи на регулярни релации.
\end{enumerate}
\end{theorem}
\begin{definition}
  \marginnote{\scriptsize На англ. Dyck. \cite[стр. 142]{hopcroft}}
  Език на Дик над езика $B_n=\{(_i,)_i\,|\, 1\le i\le n\}$ е множеството от всички
  думи, които представляват правилно разположени скоби.
\end{definition}
\begin{theorem}
Класът от контекстносвободни езици е точно класът от езици от вида:
\begin{equation*}
h(D_n \cap R),
\end{equation*}
където $D_n$ е език на Дик, $R$ е регулярен език, а $h:B_n^*\rightarrow\Sigma^*$ е хомоморфизъм.
\end{theorem}
\begin{definition}
За азбука $\Sigma=\{a_1,a_2,\dots,a_n\}$ и дума $w\in \Sigma^*$ с $|w|_i$ бележим броя букви $a_i$ в $w$.
Дефинираме $\|w\|=(|w|_1,|w_2,\dots,|w|_n)\in \mathbb{N}^n$.
\end{definition}
\begin{definition}
За вектор $u\in \mathbb{N}^n$ и множество от вектори $V\subseteq \mathbb{N}^n$ с $u+ V$ и $V^*$ бележим множествата от вектори:
\begin{eqnarray*}
u+V = \{u+v\,|\, v\in V\} \text{ и } V^*=\{\sum_{i=1}^k d_iv_i\,|\, k\in \mathbb{N}, d_i\in \mathbb{N}\text{ и } v_i\in V\}.
\end{eqnarray*}
\end{definition}
\begin{theorem}[на Парих] За всеки контекстносвободени език $L\subseteq \{a_1,a_2,\dots,a_n\}^*$ има
естествено число $k$, вектори $u_1,u_2,\dots,u_k\in \mathbb{N}^n$ и крайни множества $V_1,V_2,\dots,V_k\subseteq \mathbb{N}^n$,
за които:
\begin{equation*}
\{\|w\|\,|\, w\in L\} = \bigcup_{i=1}^k (u_i + V_i^*).
\end{equation*}
\end{theorem}
Пучително е да се разгледа частния случай на еднобуквена азбука. Тогава, тъй като $w=a_1^{|w|_1}$ и $\|w\|=(|w|_1)$,
може да отъждествим $w$ с $\|w\|$. Освен това в този случай може да предполагаме, че $|V_i|=1$ за всяко $V_i$ от теоремата
на Парих (може да вземем най-малкото общо кратно в по-големите множества и да добавим повече вектори $u$). Тогава
$V_i=\{v_i\}$ и изразите $u_i+V_i^*=\{u_i+dv_i \,|\, d\in \mathbb{N}\}$ представляват аритметични прогресии. 

Тоест, ако се абстрахираме от реда на буквите, контекстносвободните езици представляват крайни обединения на аритметични
прогресии.


\section{$\texttt{PSPACE}$ и $\texttt{NPSPACE}$ проблеми}

\begin{proposition}
  Езикът
  \[L = \{\code{\A} \mid \A \text{ е НКА и }\L(\A) = \Sigma^\star\}\]
  е $\texttt{PSPACE}$-пълен.
\end{proposition}

\begin{theorem}[Савич 1970]
  \[\texttt{NPSPACE}(f(n)) \subseteq \texttt{PSPACE}(f^2(n)).\]
\end{theorem}

\begin{corollary}
  $\texttt{NPSPACE} = \texttt{PSPACE}$.
\end{corollary}

\begin{theorem}
  Езикът
  \[L = \{ \code{\A_1} \sharp \code{\A_2} \mid \A_1,\A_2\text{ са НКА и }\L(\A_1) = \L(\A_2)\}\]
  е $\texttt{PSPACE}$-пълен.
\end{theorem}

\begin{theorem}
Проблемът ($\texttt{INT}$):
\begin{alltt}
Вход: \(n\in\mathbb{N}, A\sb{i}=\langle\Sigma,Q\sb{i},s\sb{i},\delta\sb{i},F\sb{i}\rangle\) - КДА
Въпрос: \(\bigcap\sb{i}\sp{n}L(A\sb{i})=\Sigma\sp{*}\)?
\end{alltt}
е $\texttt{PSPACE}$-пълен.
\end{theorem}

\begin{theorem}
Проблемът ($\texttt{MIN(NFA}\rightarrow\texttt{NDA)}$):
\begin{alltt}
Вход: \(k\in\mathbb{N}, A=\langle\Sigma,Q,s,\delta,F\rangle\) - КДА
Въпрос: има ли НДА с език \(L(A)\) и не повече от \(k\) състояния?
\end{alltt}
е $\texttt{PSPACE}$-пълен.
\end{theorem}

\begin{theorem}[Immerman-Szelepcs\'enyi]
  \[\texttt{NPSPACE}(S(n)) = \texttt{co-NPSPACE}(S(n)).\]
\end{theorem}

\begin{corollary}
  Допълнението на всеки контекстнозависим език е контекстно зависим.
\end{corollary}


\section{$\texttt{NP}$ проблеми}

% \begin{theorem}
%   Ако $L \in \texttt{NP}$, то $L \in \texttt{EXP}$.
% \end{theorem}

\begin{theorem}
  \marginnote{\scriptsize \cite[стр. 262]{papadimitriou-lewis}}
  % \marginnote{\scriptsize Дори не е полуразрешим}
  Домино проблемът е неразрешим.
\end{theorem}
\begin{hint}
  Свеждаме ефективно езика
  \[\{\code{\M} \mid \M\text{ работи безкрайно дълго при вход }\varepsilon\}\]
  към домино проблема.
\end{hint}

\begin{theorem}
  \marginnote{\scriptsize \cite[стр. 310]{papadimitriou-lewis}}
  Ограниченият домино проблем е $\texttt{NP}$-пълен
\end{theorem}

\begin{theorem}[Кук]
  \marginnote{\scriptsize \cite[стр. 312]{papadimitriou-lewis}}
  $\texttt{SAT}$ проблемът е $\texttt{NP}$-пълен.
\end{theorem}
\begin{hint}
  Ограниченият домино проблем се свежда полиномиално към $\texttt{SAT}$.
\end{hint}

\begin{theorem}
  \marginnote{\scriptsize \cite[стр. 313]{papadimitriou-lewis}}
  $\texttt{3-SAT}$ проблемът е $\texttt{NP}$-пълен.
\end{theorem}
\begin{hint}
  $\texttt{SAT}$ се свежда полиномиално към $\texttt{3-SAT}$.
\end{hint}

За един неориентиран граф $G = (V,E)$,
казваме, че множеството $C \subseteq V$ {\bf покрива върховете} на $G$, ако
\[(\forall (u,v) \in E)[u \in C\ \lor\ v \in C].\]
$\texttt{Vertex Cover}$ е проблемът да се определи дали по даден неориентиран граф $G$ и $k \in \Nat$
съществува покритие $C$ на върховете на $G$, за което $|C| = k$.

\begin{theorem}
  \marginnote{\scriptsize \cite[стр. 331]{hopcroft}}
  $\texttt{Vertex Cover}$ проблемът е $\texttt{NP}$-пълен.
\end{theorem}
\begin{hint}
  $\texttt{3-SAT}$ се свежда полиномиално към $\texttt{Vertex Cover}$.
\end{hint}


\subsection*{$\texttt{NP}$-пълни проблеми в теория на езиците}

\begin{definition}
Казваме, че недетреминиран краен автомат $A=\langle \Sigma,Q,\{s\},\Delta,F\rangle$ е еднозначен,
ако $\Delta\subseteq Q\times \Sigma\times Q$ и за всяка дума $w\in L(A)$ има точно един успешен път с етикет $w$ в $A$.
\end{definition}

\begin{lemma}
  Ако $A_i=\langle \Sigma,Q_i,\{s_i\},\Delta_i,F_i\rangle$ за $i=1,2$ са еднозначни НДА, то следните са еквивалентни:
  \begin{enumerate}
  \item $\exists w\in L(A_1)\triangle L(A_2) (|w|\le |Q_1|+|Q_2|)$.
  \item $L(A_1)\triangle L(A_2)\neq\emptyset$.
  \end{enumerate} 
\end{lemma}

\begin{theorem}
Проблемът ($\texttt{MIN(UFA}\rightarrow\texttt{UFA)}$):
\begin{alltt}
Вход: \(k\in\mathbb{N}, A=\langle\Sigma,Q,s,\delta,F\rangle\) - еднозначен НДА
Въпрос: има ли еднозначен НДА с език \(L(A)\) и не повече от \(k\) състояния?
\end{alltt}
е $\texttt{NP}$-пълен.
\end{theorem}

\begin{theorem}[Ladner]
  \marginnote{\scriptsize \cite[стр. 330]{papadimitriou}}
  Ако $\texttt{P} \neq \texttt{NP}$, то съществува $\texttt{NP}$ проблем, който не е $\texttt{P}$ проблем и не е $\texttt{NP}$-пълен проблем.
\end{theorem}

\section{Машини на Тюринг с оракул}

\begin{theorem}[Baker-Gill-Solovay]
  \marginnote{\scriptsize \cite[стр. 340]{papadimitriou}}
  Съществуват оракули $A$ и $B$, за които $\texttt{P}^A \neq \texttt{NP}^A$ и $\texttt{P}^B = \texttt{NP}^B$.
\end{theorem}



\section{Задачи}


\begin{problem}
  \marginnote{\scriptsize \cite[стр. 184]{papadimitriou}}
  Докажете, че $\texttt{2-SAT}$ проблемът е $\texttt{PTIME}$.
\end{problem}

\begin{problem}
  Докажете директно, че $\texttt{SAT}$ е $\texttt{NP}$-труден проблем
  като кодирате историята на едно изчисление на недетерминистична машина на Тюринг
  с формула в конюнктивна нормална форма.
\end{problem}

\begin{problem}
  Докажете, че $\texttt{Exact Cover}$ е $\texttt{NP}$-пълен проблем.
\end{problem}

\begin{problem}
  Докажете, че $\texttt{Hamilton Cycle}$ е $\texttt{NP}$-пълен проблем.
\end{problem}
\begin{hint}
  $\texttt{3-SAT}$ се свежда полиномиално към $\texttt{Hamilton Cycle}$.
\end{hint}



\bibliographystyle{amsalpha}
\bibliography{mesi}


\end{document}

%%% Local Variables:
%%% mode: latex
%%% TeX-master: t
%%% End:
